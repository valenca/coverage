\cleardoublepage
\chapter{Conclusion}
\label{chap:conc}
\lhead{Chapter \ref{chap:conc}. \emph{\nameref{chap:conc}}}
\vspace{-15pt}
In this thesis, we designed algorithms to select representative subsets from large quantities of geographic points. The algorithm must be able to handle panning and zooming motions along the geographic region displayed, as well as be efficient enough to be integrated in the back-end of a real-time Web application. 

We started by defining representation as finding the optimal solution to the \emph{$k$-centre} problem. To solve this problem, two different incremental branch-and-bound approaches were implemented. We then compared the performance of these approaches and one formulation of the problem in integer linear programming to ascertain if the problem could be solved in real-time. The results showed not only that optimal approaches are not efficient to meet the time requirements of the application, but also that the problem required a priori information about the number of point clusters in the region.

In a second phase, we redefine the representation problem as to find the solution to the \emph{geometric disk cover} problem instead. This new approach manages to compute the cardinality of the final subset with no information about the region. To solve this problem, we implemented two versions of an approximation algorithm, both of which calculated subsets very efficiently, with a guarantee of approximation to the optimal value. We then performed some tests to determine the most efficient solution. Additionally, we used heuristic approaches to further speed up the algorithm and reduce its memory usage, but without guarantee of approximation.

Throughout this thesis, we found that interpreting the representation problem as \emph{geometric disk cover} problem, and solve it with an approximation algorithm is an efficient way to be applied in the context of a real-time web application. We also found viable solutions for solving larger instances of the problem.

\section{Future Work}
The next step for the web application developer is to integrate the algorithms described and implemented during this thesis. Since the algorithms are already implemented taking into account their place in the architecture of the final product, the inputs and outputs are already conformed to the specifications given. Full integration requires that the communication layer, as well as the proper protocol request and response parsers are implemented.

Although the results were satisfactory, it may still be possible to develop more efficient algorithms. The bottleneck for our final approaches lies in the proximity graph building stage, particularly performing the various range search queries to establish neighbouring points. This means that implementing more efficient range search structures is a good strategy to find better solutions. Furthermore, since the evaluation of the results may depend of the perspective of the user, new solutions might require different interpretations on the concept of representation, such as the notion of uniformity or other similar metrics. 
