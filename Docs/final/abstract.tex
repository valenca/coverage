\setcounter{page}{0}
\pagenumbering{roman}
\vspace*{0.1cm}
\section*{\huge Abstract}

In recent years, Geographic Information Systems have witnessed a large increase in data availability. There is a need to process a large amount of data before it can be managed and analysed. This project aims to develop an application operating through a Web platform in order to allow for a low cost and simplified integration, management and manipulation of georeferenced information. Special emphasis is given to the implementation of efficient clustering algorithms for finding a representative set of points in a map. In the thesis, this representation problem is formulated as two classic optimisation problems: the \emph{$k$-center} and the \emph{geometric disk cover}. The approaches covered in this thesis include exact algorithms for solving the \emph{k-centre} problem, as well as approximation algorithms and heuristic methods to solve the \emph{geometric disk cover} problem. The algorithms are experimentally evaluated in a wide range of scenarios.

\subsection*{\large Keywords}

Geographic Clustering, Computational Geometry Algorithms, Coverage Problems, Real-Time Applications.

\vspace*{0.6cm}

\section*{\huge Resumo}

Nos últimos anos, os Sistemas de Informação Geográfica têm assistido a um grande aumento na quantidade de dados disponíveis. De facto, existe uma necessidade de encontrar uma maneira eficiente de processar grandes quantidades de dados para que tanta informação possa ser facilmente gerida e analisada. Este projeto visa desenvolver uma aplicação para uma plataforma Web, de modo a obter uma integração simples e de baixo custo que manipule e analise dados georeferenciados. Uma ênfase especial é dada à implementação de algoritmos para encontrar um conjunto representativo de pontos num mapa. Nesta tese, o problema de representação é formulado como dois problemas de optimisação clássicos: \emph{$k$-centre} e \emph{geometric disk cover}. As abordagens descritas nesta tese incluem algoritmos exactos para resolver o problema do \emph{k-centre}, bem como algoritmos de aproximação e métodos heurísticos para resolver o problema de \emph{geometric disk cover}. Os algoritmos são experimentalmente avaliados numa grande seleção de cenários.

\subsection*{Palavras-chave}

Clustering Geográfico, Algoritmos de Geometria Computacional, Problemas de Cobertura, Aplicações em Tempo Real.

\vfill