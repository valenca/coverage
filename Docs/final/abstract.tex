\setcounter{page}{0}
\pagenumbering{roman}

\section*{\huge Abstract}

In recent years, Geographic Information Systems (GIS) have witnessed a large increase in data availability. There is a need to process a large amount of data before it can be managed and analysed. This project aims to develop a GIS application operating through a Web platform in order to allow for a low cost and simplified integration, management and manipulation of georeferenced information. Special emphasis is given to the implementation of efficient clustering algorithms for finding a representative set of points in a map, which can be recast as a \emph{k-center} problem. The approaches covered in this report include exact algorithms for finding minimum coverage subsets.

\subsection*{\large Keywords}

Geographic Clustering, k-Center, Coverage, Branch-and-Bound, Delaunay Triangulations

\section*{\huge Resumo}

Nos últimos anos, os Sistemas de Informação Geográfica (GIS) têm assistido a um grande aumento na quantidade de dados disponíveis. De facto, existe uma necessidade de encontrar uma maneira eficiente de processar grandes quantidades de dados para que tanta informação possa ser facilmente gerida e analisada. Este projeto visa desenvolver uma aplicação GIS para uma plataforma Web, de modo a obter uma integração simples e de baixo custo que manipule e analise dados georeferenciados. Uma ênfase especial é dada à implementação de algoritmos para encontrar um conjunto representativo de pontos num mapa, que pode ser formulado como um problema de \emph{k-center}. As abordagens descritas neste relatório incluem algoritmos exactos para encontrar o subconjunto ótimo.

\subsection*{Palavras-chave}

Clustering Geográfico, k-Center, Cobertura, Branch-and-Bound, Triangulações Delaunay

